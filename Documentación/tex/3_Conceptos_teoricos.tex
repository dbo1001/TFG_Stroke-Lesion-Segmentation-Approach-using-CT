\capitulo{3}{Conceptos teóricos}

Este proyecto aplica la Ingeniería Informática en el ámbito de la Biomedicina, colaborando, más concretamente, con el diagnóstico por medio de imágenes en Neurología. Esto implica un proceso de investigación y aprendizaje necesarios para el entendimiento del entorno, lo que supone mayor comprensión del vocabulario y objetivos específicos, así como para el desarrollo del proyecto en sí.

En este caso, nos enfocamos en conceptos básicos bastante concretos, dentro de dos áreas de conocimiento:
\begin{itemize}
	\item Neurología
	\item Técnicas de Imagen Biomédica
\end{itemize}

En el primer caso, los conceptos teóricos neurológicos que nos interesan son los referentes al accidente (o ataque) cerebro-vascular (ACV), más frecuentemente denominado ictus, y el tejido isquémico que resulta del mismo.

Por otro lado, dentro de las Técnicas de Imagen Biomédica, nos centramos en las Tomografías (Axiales) Computarizadas, conocidas comúnmente como TC o TAC. Principalmente, debemos tener claro el concepto de TC craneal, ya que es el tipo de imagen para el que este proyecto ha sido diseñado. Sin embargo, no está de más conocer por encima algo sobre las Resonancias Magnéticas (MRI por sus siglas en inglés \textit{Magnetic Resonance Imaging}), ya que su comparativa con las Tomografías Computarizadas dan más sentido e importancia a este proyecto.


\section{Accidente Cerebrovascular o Ictus}

El accidente (o ataque) cerebrovascular (ACV) es más comúnmente conocido como ictus, pero también mediante otras expresiones como infarto o derrame cerebral, embolia, trombosis o apoplejía \cite{wiki:acv}. No se trata de un uso preciso de dichos términos, ya que los últimos mencionados no son sinónimos, solo que pueden (o no) estar frecuentemente relacionados con la causa o tipo de ictus. Por ejemplo, el hecho de referirse a esta enfermedad cerebrovascular como ``infarto'' frente a ``derrame'' o ``hemorragia'' cerebral, se explica por los dos tipos en que se presenta:
\begin{description}
	\item [Ictus isquémico]: causado por una disminución importante en el flujo sanguíneo de una parte del cerebro, ante la oclusión de una arteria, provocando que el tejido cerebral de la zona no reciba la sangre y oxígeno necesarios para su normal funcionamiento. Se calcula que alrededor del 87\% de los ictus son isquémicos \cite{wiki:stroke}.	
	\imagen{ictus_isquemico}{Esquema representativo de un ictus isquémico.}
	
	\item [Ictus hemorrágico]: debido a la rotura de un vaso cerebral, resultando en un sangrado que aumenta la presión sobre el cerebro en esa zona, afectando a su correcto funcionamiento. En algún caso, se pueden combinar, sin saber con certeza cuántos ictus hemorrágicos comenzaron como isquémicos, ya que se puede producir un sangrado en zonas de isquemia \cite{wiki:stroke}.
	\imagen{ictus_hemorragico}{Esquema representativo de un ictus hemorrágico.}
\end{description}



\section{Tomografía Computarizada (TC/TAC)}




\subsection{TC o TAC craneal}



\imagen{ictus_TC}{Imágenes de TC craneal mostrando lesiones de ictus isquémico (A) y hemorrágico (B). Se ve el tejido isquémico oscuro y el sangrado blanco.}


\subsection{Alternativa a la Resonancia Magnética (MRI)}





\indexspace
\indexspace
\indexspace
\indexspace
Algunos conceptos teóricos de \LaTeX \footnote{Créditos a los proyectos de Álvaro López Cantero: Configurador de Presupuestos y Roberto Izquierdo Amo: PLQuiz}.


\subsubsection{Subsubsecciones}

Y sub - sub - secciones. 


\section{Listas de items}

Existen tres posibilidades:

\begin{itemize}
	\item primer item.
	\item segundo item.
\end{itemize}

\begin{enumerate}
	\item primer item.
	\item segundo item.
\end{enumerate}

\begin{description}
	\item[Primer item] más información sobre el primer item.
	\item[Segundo item] más información sobre el segundo item.
\end{description}
	
\begin{itemize}
\item 
\end{itemize}

\section{Tablas}

Igualmente se pueden usar los comandos específicos de \LaTeX o bien usar alguno de los comandos de la plantilla.

\tablaSmall{Herramientas y tecnologías utilizadas en cada parte del proyecto}{l c c c c}{herramientasportipodeuso}
{ \multicolumn{1}{l}{Herramientas} & App AngularJS & API REST & BD & Memoria \\}{ 
HTML5 & X & & &\\
CSS3 & X & & &\\
BOOTSTRAP & X & & &\\
JavaScript & X & & &\\
AngularJS & X & & &\\
Bower & X & & &\\
PHP & & X & &\\
Karma + Jasmine & X & & &\\
Slim framework & & X & &\\
Idiorm & & X & &\\
Composer & & X & &\\
JSON & X & X & &\\
PhpStorm & X & X & &\\
MySQL & & & X &\\
PhpMyAdmin & & & X &\\
Git + BitBucket & X & X & X & X\\
Mik\TeX{} & & & & X\\
\TeX{}Maker & & & & X\\
Astah & & & & X\\
Balsamiq Mockups & X & & &\\
VersionOne & X & X & X & X\\
} 

